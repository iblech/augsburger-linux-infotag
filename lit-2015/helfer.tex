\documentclass[a4paper,ngerman,14pt]{scrartcl}

\usepackage[utf8]{inputenc}

\usepackage[ngerman]{babel}
\usepackage{latexsym}

\usepackage[protrusion=true,expansion=true]{microtype}

\usepackage[T1]{fontenc}
\usepackage{libertine}
\usepackage{tabto}
\usepackage{setspace}

\setlength\parskip{\medskipamount}
\setlength\parindent{0pt}

\usepackage{geometry}
\geometry{tmargin=1cm,bmargin=1cm,lmargin=2cm,rmargin=2cm}

\pagestyle{empty}

\begin{document}

\begin{center}
  \Huge\sffamily
  {\textbf{Linux-Infotag}}
\end{center}

\section*{Allgemeine Hinweise}

\begin{itemize}
  \item Der Vortrag sollte nicht länger als 45 Minuten dauern,
  damit danach noch Zeit für Fragen (fünf bis zehn Minuten) und Raumwechsel ist.
  Die Vortragenden freuen sich mit den Schildern über Zeithinweise
  vor Ablauf der 45 Minuten.

  \item Damit Live-Beispiele in der Konsole auch hinten im Raum lesbar sind,
  sollte die Schrift groß genug (mindestens Schriftgröße~20) und schwarz auf
  weißem Hintergrund sein. Bitte unbedingt darauf achten und ggf. den
  Vortragenden bitten, die Schrift zu vergrößeren (geht oft ganz einfach:
  Strg+Plus).

  \item Wenn die Verbindung zwischen Laptop und Beamer nicht funktioniert:
  \begin{enumerate}
    \item Entsprechende Funktionstasten am Laptop drücken
    \item \texttt{xrandr -{}-output VGA1 -{}-auto}
    \item \texttt{xrandr -{}-output VGA1 -{}-mode 1024x768}
    \item X-Server neustarten
    \item Laptop neustarten
  \end{enumerate}
\end{itemize}


\section*{Vortragsevaluation}

\newcommand{\longline}{\underline{\hspace{12cm}}}
\newcommand{\shortline}{\underline{\hspace{3cm}}}

\newcommand{\checkbox}{$\ \Box\ $}
\newcommand{\boxes}{%
  \checkbox \> \checkbox \> \checkbox \> \checkbox \> \checkbox}

\onehalfspacing

\begin{tabbing}
  \textbf{Beginn des Vortrags:}\ \ \ \=
  (sehr unverständlich) \=
  \checkbox \= \checkbox \= \checkbox \= \checkbox \= \checkbox \=
  \kill

  \textbf{Titel des Vortrags:} \> \longline \\
  \textbf{Beginn des Vortrags:} \> \shortline{} Uhr \\\\

  \textbf{Raumbetreuer:} \> \longline \\\\

  \textbf{Anzahl Zuhörer:} \> \shortline \\\\

  \textbf{Verständlichkeit:} \> (sehr unverständlich) \> \boxes \> (sehr
  verständlich) \\
  \textbf{Inhalt:} \> (sehr langweilig) \> \boxes \> (sehr interessant) \\
  \textbf{Humor:} \> (sehr trocken) \> \boxes \> (sehr lustig) \\
  \textbf{Vortragsstil:} \> (sehr schlecht) \> \boxes \> (sehr gut) \\
  \textbf{Publikumsfragen:} \> etwa \shortline{} Stück \\
  \textbf{Gesamteindruck:} \> (sehr schlecht) \> \boxes \> (sehr gut) \\
  \textbf{Bemerkungen:} \>
\end{tabbing}

\end{document}
